% !TeX root = main.tex

\hypertarget{masters-degrees}{%
\section{Master's Degrees}\label{masters-degrees}}

\textbf{Master of Science (MS)} degree: a student completes the coursework
and conducts independent research under the supervision of a professor that requires
a written thesis and oral defense; the level of research effort is
commensurate with \textbf{two} typical academic courses.

\textbf{Master of Computer Science (MCS)} degree, which either focuses
on \emph{all coursework} (the student performs no independent research)
or involves a \emph{project} (student conducts independent research
overseen by a professor where the level of research effort is
commensurate with \textbf{one} typical academic course).

\textbf{Note:} A Master's degree student is assigned an academic advisor
upon entering the program. If the student selects an MS or MCS
(project) degree, their research advisor becomes the academic
advisor.

\textbf{Terminology}. In this document, ``Master's degrees'' refer to
MS, MCS (project) and MCS (coursework); ``MCS'' refers to both MCS
(project) and MCS (coursework).

\hypertarget{masters-degree-requirements}{%
\subsection{Master's Degree \& Committee
Requirements}\label{masters-degree-requirements}}

Please refer to the University Graduate Record for
degree requirements for the Master's Degrees in Computer Science:


% Magic macro to enable for loops to generate table rows.
% https://tex.stackexchange.com/a/630997/49699
\makeatletter
    \newcommand*{\@MyTempTableTokens}{}%
    \newtoks\@tabtoks
    %%% assignments to \@tabtoks must be global, because they are done in \foreach
    \newcommand\AddTableTokens[1]{\global\@tabtoks\expandafter{\the\@tabtoks#1}}
    \newcommand\eAddTableTokens[1]{%
        %% https://tex.stackexchange.com/a/175573/4301
        \protected@edef\@MyTempTableTokens{#1}%
        \expandafter\AddTableTokens\expandafter{\@MyTempTableTokens}%
    }
    %%% variable should always be operated on either always locally or always globally
    \newcommand*\ResetTableTokens{\global\@tabtoks{}}
    \newcommand*\PrintTableTokens{\the\@tabtoks}
\makeatother


% Generate rows of Graduate Record links per academic year.
\ResetTableTokens%
\foreach \year in \yearstoshow {%
    \eAddTableTokens{\csname\year\endcsname & \noexpand\makecell{\href{\csname GR\year MS\endcsname}{Computer Science, M.S.} \\ \href{\csname GR\year CR\endcsname}{Committee Requirements}} & \href{\csname GR\year MCS\endcsname}{Computer Science, M.C.S.} \\ \noexpand\hline}
}%

% Update this table based on what is defined in link.tex.
\begin{tabularx}{\textwidth}{|l|l|X|}
    \hline
    \textbf{Graduate Record} & \textbf{MS Page Link} & \textbf{MCS Page Link} \\ \hline
    \PrintTableTokens
\end{tabularx}





% fxl: mar 2025. from jwd. indeed cs faculty do not have to be majority, but it is the way
The MS thesis committee must include the following.
\begin{itemize}
  \item the thesis advisor, who must be CS faculty and supervises the student for CS 8999 (Thesis);
  \item in total, a minimum of three UVA faculty, at least two of whom must be SEAS faculty, as mandated by the Graduate Record linked above.
\end{itemize}

For MS thesis committees, CS faculty are defined as those with primary, secondary, or \underline{courtesy} appointments in CS. Note that this definition may differ from the definition of CS faculty for PhD committees.

\hypertarget{master-teaching-assistant-positions-mta}{%
\subsection{Master Teaching Assistant Positions
(MTA)}\label{master-teaching-assistant-positions-mta}}

Depending upon demand, Master's students (typically those in their
second year) may have the opportunity to apply for a limited number of
\emph{Master Teaching Assistant (MTA)} positions. An MTA is an excellent
way to gain teaching experience and supplement income. MTA positions pay
an hourly wage without any other benefits.

MTAs typically help a professor by grading assignments and exams,
holding office hours, etc. It is UVA policy that all graduate TAs whose
first language is other than English must first take the SPEAK test to
assess their fluency with oral English and to determine the types of
actions they may perform.
See Section~\ref{english-language-proficiency-assessments-written-and-oral}.

\hypertarget{en-route-masters-degree}{%
\subsection{En Route Master's
Degree}\label{en-route-masters-degree}}

A current PhD student can obtain a CS Master's degree along the way to
the PhD degree by completing all the requirements for the Master's
degree. A PhD student terminating the PhD program before graduation can
also obtain a Master's degree if he or she has completed all
requirements for the degree. In either case, please consult with the MGPD and the Graduate Student Coordinator.

\hypertarget{typical-timeline}{%
\subsection{Typical Timeline}\label{typical-timeline}}

In the description below, ``Master's students'' refer to MS, MCS
(project), and MCS (coursework); ``MCS students'' refer to
MCS (project) and MCS (coursework) students.

Most students finish within three or four semesters. However, the
official deadline to complete a degree is 5 years for an MS student and
7 years for an MCS student. In response to the COVID-19 pandemic,
students registered in the MS program in Spring 2020 have a deadline of
6 (rather than 5) years; the deadline for MCS students is unchanged.

UVA policy is that all students whose first language is other than
English who wish to serve as an MTA or a GTA must take the UVELPE Oral
(formally known as SPEAK) test upon arrival.

{First semester (Fall):} All first-semester Master's students take CS
6190 (Computer Science Perspectives) for 1 credit and 4 graded, 3-credit
graduate courses (total 13 credits). MS and MCS (project) students should use this semester to explore departmental research projects, discuss potential research ideas with faculty advisors, and finalize their research plans.

{Second semester (Spring):} MCS students typically take 4 graded courses
(12 credits). MS students typically take 3 graded classes and 3 credits
of CS 8999 (Thesis) with their thesis advisor.

{Summer:} Students typically are away on internships, potentially
facilitated
\href{https://engineering.virginia.edu/about/offices/center-engineering-career-development}{through
the SEAS \emph{Center for Engineering Career Development}}.
International students use the Curricular Practical Training program (CPT) organized
by the UVA \emph{International Studies Office} (ISO) and the
\emph{International Students and Scholars Program} (ISSP).

{Third semester:} Master's students typically take the remaining courses
necessary to fulfill their academic requirements. An MS student
typically takes 3 credits of CS 8999 (Thesis) with the research advisor
in this semester (in addition to the 3 credits of CS 8999 taken in the
previous Spring semester). An MCS (project) student typically takes CS
7995 (Supervised Project Research) with the research advisor in this
semester.

Students must use SIS to apply for graduation with a Master's degree at
the start of the semester during which they expect to graduate --
usually no later than 1 October (Fall), 1 February (Spring), or 1 June
(Summer). Students should check their completion of the requirements
using the Academic Requirements tool within the \emph{Student
Information System}~(SIS). Two special situations may also occur toward
the end of Master's studies:

\begin{enumerate}
\def\labelenumi{\arabic{enumi}.}
\item
If a Master's candidate decides to defer graduation for another semester, they should seek advice and assistance from the MGPD.
\item
  If a Master's student wishes to pursue a PhD in UVA CS:
  (1) the student should first identify a CS professor and then confirm that he or she is willing to fund and advise the student's PhD research;
  (2) With a confirmed advisor, the student should then consult with the Chair of the
  CS Graduate Admissions Committee (see Section 1) to determine the
  necessary forms and procedures used to apply to our PhD program.
  (3) Once application forms have been filed with the Graduate Coordinator, the CS Graduate Admissions
  Committee will determine whether to offer admission to the PhD
  program.
\end{enumerate}