% !TeX root = main.tex

\hypertarget{phd-degree}{
\section{Ph.D. Degree}
\label{phd-degree}
}

To obtain a Ph.D. degree, several steps are required, including:

\begin{enumerate}[nosep]
    \item Completing required coursework (see \cref{phd-coursework}).
    \item Transferring credits from another university, if applicable (see \cref{phd-transfer-credit}).
    \item Finding an advisor through the first-year rotation program (see \cref{first-year-phd-rotation-program}).
    \item Passing the Ph.D. qualifying exam (see \cref{phd-qualifying-examination}).
    \item Forming a Ph.D. committee (see \cref{phd-committee}).
    \item Writing and presenting a Ph.D. proposal (see \cref{phd-proposal}).
    \item Writing a Ph.D. dissertation (see \cref{phd-dissertation}).
    \item Presenting a public defense of the dissertation (see \cref{phd-defense}).
    \item Completing four semesters of part-time work (or equivalent) as a Graduate Teaching Assistant (GTA).
\end{enumerate}

A typical timeline for completing a Ph.D. is described in \Cref{typical-timeline-phd},
while financial support options are discussed in \Cref{first-year-fellowship-gras-and-gtas}.

\textbf{Terminology}. A person who has an undergraduate degree and who
wishes to pursue a Ph.D. is known as \emph{Ph.D. student} or
\emph{doctoral student}. That person advances to the status of \emph{PhD
candidate} or \emph{doctoral candidate} after completing all the coursework
and passing the Ph.D. Qualification Examination.

\hypertarget{phd-graduate-record}{%
\subsection{Graduate Record Links}
\label{phd-graduate-record}
}

The following links lead to the relevant pages of the Graduate Record, which
outlines UVA's official procedures and policies. The Graduate Record is updated
annually, potentially including minor adjustments.

\textbf{2025-2026}

\begin{tabularx}{\textwidth}{|l|X|}
    \hline
        \textbf{Topic} & \textbf{Program Page Link} \\ \hline
        \makecell{Coursework \& Program Requirements} & \makecell{\href{\GRtwentyfivePHD}{Computer Science, Ph.D.}} \\ \hline
        \makecell{Transfer Credit}                    & \makecell{\href{\GRtwentyfiveTC}{Transfer Credit}}          \\ \hline
        \makecell{Breadth Courses}                    & \makecell{\href{\GRtwentyfiveBC}{Computer Science, Ph.D.}}  \\ \hline
        \makecell{Committee Requirements}             & \makecell{\href{\GRtwentyfiveCR}{Committee Requirements}}   \\ \hline
\end{tabularx}


\textbf{2024-2025} (archived)

\begin{tabularx}{\textwidth}{|l|X|}
    \hline
        \textbf{Topic} & \textbf{Program Page Link} \\ \hline
        \makecell{Coursework \& Program Requirements} & \makecell{\href{https://records.ureg.virginia.edu/preview_program.php?catoid=62\&poid=9192}{Computer Science, Ph.D.}} \\ \hline
        \makecell{Transfer Credit}                    & \makecell{\href{https://records.ureg.virginia.edu/content.php?catoid=62\&navoid=5418\#transfer-credit}{Transfer Credit}} \\ \hline
        \makecell{Breadth Courses}                    & \makecell{\href{https://records.ureg.virginia.edu/preview_program.php?catoid=62\&poid=9192}{Computer Science, Ph.D.}} \\ \hline
        \makecell{Committee Requirements}             & \makecell{\href{https://records.ureg.virginia.edu/content.php?catoid=62\&navoid=5418\#committee-requirements}{Committee Requirements}} \\ \hline
\end{tabularx}

\textbf{2023-2024} (archived)

\begin{tabularx}{\textwidth}{|l|X|}
    \hline
        \textbf{Topic} & \textbf{Program Page Link} \\ \hline
        \makecell{Coursework \& Program Requirements} & \makecell{\href{https://records.ureg.virginia.edu/preview_program.php?catoid=57\&poid=8696}{Computer Science, Ph.D.}} \\ \hline
        \makecell{Breadth Courses}                    & \makecell{\href{https://records.ureg.virginia.edu/preview_program.php?catoid=57\&poid=8696}{Computer Science, Ph.D.}} \\ \hline
        \makecell{Committee Requirements}             & \makecell{\href{https://records.ureg.virginia.edu/content.php?catoid=57\&navoid=5188\#committee-requirements}{Committee Requirements}} \\ \hline
\end{tabularx}

\textbf{2022-2023} (archived)

\begin{tabularx}{\textwidth}{|l|X|}
    \hline
        \textbf{Topic} & \textbf{Program Page Link} \\ \hline
        \makecell{Coursework \& Program Requirements} & \makecell{\href{https://records.ureg.virginia.edu/preview_program.php?catoid=55\&poid=7381}{Computer Science}} \\ \hline
\end{tabularx}

\textbf{2021-2022} (archived)

\begin{tabularx}{\textwidth}{|l|X|}
    \hline
        \textbf{Topic} & \textbf{Program Page Link} \\ \hline
        \makecell{Coursework \& Program Requirements} & \makecell{\href{https://records.ureg.virginia.edu/preview_program.php?catoid=53\&poid=6994}{Computer Science}} \\ \hline
\end{tabularx}

\textbf{2020-2021} (archived)

\begin{tabularx}{\textwidth}{|l|X|}
    \hline
        \textbf{Topic} & \textbf{Program Page Link} \\ \hline
        \makecell{Coursework \& Program Requirements} & \makecell{\href{https://records.ureg.virginia.edu/preview_program.php?catoid=50\&poid=6470}{Computer Science}} \\ \hline
\end{tabularx}

\textbf{2019-2020} (archived)

\begin{tabularx}{\textwidth}{|l|X|}
    \hline
        \textbf{Topic} & \textbf{Program Page Link} \\ \hline
        \makecell{Coursework \& Program Requirements} & \makecell{\href{https://records.ureg.virginia.edu/preview_program.php?catoid=48\&poid=6159}{Computer Science}} \\ \hline
\end{tabularx}





\hypertarget{phd-coursework}{%
\subsection{Ph.D. Coursework}\label{phd-coursework}}

Please refer to the Graduate Record for Ph.D. degree requirements.

\hypertarget{phd-transfer-credit}{
\subsection{\texorpdfstring{Ph.D. Transfer Credit}{Ph.D. Transfer Credit}}
\label{phd-transfer-credit}
}

If a graduate student enters the CS Ph.D. program \emph{with a Master's
degree in a computing field}, they will receive 24 bulk transfer
credits.

\begin{itemize}
\item At least 6 additional credits of graded, graduate-level CS coursework must
 be taken at UVA (i.e., they cannot be transferred). A minimum grade of ``B''
 is required.
\end{itemize}

If entering \emph{without a Master's degree}, then a maximum of 6
credits of graded, graduate-level computer science coursework may be transferred.
Transferred credits must not have been used to fulfill the requirements
of any other degree.

Whether an individual transfer course counts toward the Qualifying Exam Breadth
Requirement is determined by the PGPD. For such requests, the student must
provide the syllabus of the original coursework. The Graduate Program
Coordinator facilitates this process via the Ph.D. Internal Transfer Form.
Students are encouraged to take additional courses beyond those required for
graduation.

Please refer to the Graduate Record (\Cref{phd-graduate-record}) for more information.


% !TeX root = main.tex


\hypertarget{ph.d.-student-assessment-policy-and-process}{%
\subsection{Ph.D. Student Assessment Policy and Process}\label{ph.d.-student-assessment-policy-and-process}}

\subsubsection{Goal}

The Department of Computer Science is committed to maintaining a strong
and nurturing Ph.D. degree program. To achieve this, the UVA CS Department evaluates the progress of each Ph.D. student. The purpose of this assessment is to provide both the student and their advisor with feedback that will help ensure the student's timely completion of the Ph.D. and support the achievement of their professional goals.

\subsubsection{Process Overview}

Ph.D. students will have their progress and achievements assessed twice a year by their faculty advisor and a three-member faculty committee. The results of the committee's assessment will be shared with both the student and the faculty advisor.

\subsubsection{Ph.D. Student Assessment}

A Ph.D. student assessment committee consists of three faculty members. Students will be randomly assigned to committees, ensuring they are not paired with their advisor or co-advisor. The review load will be evenly distributed among the committees.

Student progress will be evaluated twice a year, at the end of the Spring and Fall terms. 

During each assessment period, the student will fill out a self-assessment form, while their advisor will complete a corresponding assessment form for the student. The student’s assessment committee will review both the student's self-assessment and the advisor's assessment before completing the committee’s assessment form.

The committee and student reports will be submitted to the faculty advisor for feedback. Based on any comments from the faculty advisor, the committee may choose to revise their assessment report. Once the assessment report is finalized, the advisor will meet with the student to review the committee's assessment and discuss plans to address any concerns raised in the report, as well as plans for the next assessment period. 

Please note: To avoid any potential mixed messages and confusion, only the committee's assessment will be provided to the student. However, the advisor may choose to share their own assessment report with the student.

\subsubsection{Assessments of Concern}

In the case of assessments that require further attention, 
the Ph.D. Graduate Program Director (PGPD) will work with the student and their advisor to determine the
appropriate course of action. The PGPD will
also notify the Department Chair of any assessments of concern and
recommended actions.

\subsubsection{Confidential Information}

If a student desires to provide information that is not to be
shared with the advisor, 
the student can provide such information via this feedback form: 

\url{https://app.smartsheet.com/b/form/b5cd18894d8d4479a9696ca07530faa3}

This information will be made available only to the student's assessment committee, the PGPD, the CS Graduate Coordinators, and the Department Chair.
They must not disclose this information to the student's advisor or use it in the student's committee assessment, which will eventually be visible to the advisor.

If the student is an advisee of the Department Chair or the PGPD, any confidential information provided by the student will not be shared with them.



\hypertarget{first-year-phd-rotation-program}{
\subsection{First-Year Ph.D. Rotation Program}
\label{first-year-phd-rotation-program}
}

\subsubsection{Eligibility}

Entering Ph.D. students(excluding students who transfer into the CS Ph.D.
program from a different graduate program in CS, from another UVA department,
or from another institution's graduate program) shall participate in the
rotation program; they are typically supported by a First-Year Fellowship,
which provides funding for their Fall and Spring semesters during the first
year.

\subsubsection{Goals}

Students should use the rotation program to:
\begin{enumerate}[nosep]
  \item Engage in research with CS faculty to establish a permanent PhD advisor-advisee relationships.
  \item Build a solid foundation of knowledge by taking graduate-level courses.
  \item Fulfill any English as a Second Language (ESL) course requirements
\end{enumerate}

Students are expected to balance research activities and coursework; however,
they \textbf{should not reduce their coursework} to prioritize research.

% There are no other duties assigned to these fellowship students.

\subsubsection{Procedure}

The rotation program includes two rotations: one in the fall and one in the
spring semester. In each rotation, the student conducts a one-semester research
project under the supervision of a faculty member (referred to as a
\textit{rotation advisor}, see details below).

A typical rotation schedule is as follows:

\begin{itemize}[nosep]
\item \textbf{Proposal.}
  Early in the rotation (by no later than week 4), students must engage in discussions with their rotation advisor and submit a 2-page proposal. This proposal should include the following elements:
  \begin{itemize}[nosep]
    \item A description of the project and its objectives.
    \item An explanation of the project’s importance.
    \item A discussion of its novelty compared to prior work addressing similar problems.
    \item An outline of how the project's success will be evaluated.
  \end{itemize}
\item \textbf{Execution.}
  During the middle weeks of the rotation, students
  should focus on executing the activities outlined in their proposal. It is
  essential for students to schedule regular meetings with their rotation
  advisor and/or members of the advisor's research group for design discussions,
  contingency planning, and other related activities.
\item \textbf{Report.}
  By the end of the rotation (by no later than week 9),
  students are required to prepare a final report that is 4-5 pages in length,
  formatted in a style suitable for workshop or conference submissions.
  Additionally, students must deliver a brief oral presentation of their
  findings to the rotation advisor and/or members of the research group.
\end{itemize}

\textbf{Rotation 2 Proposal with Advisor of Rotation 1.}
Before the end of Rotation 1, if a student wishes to continue working with the
same rotation advisor in Rotation 2, they should prepare a new 2-page proposal
for the advisor. This proposal may draw on the experiences and results from the
current rotation, but doing so is not a requirement.


\subsubsection{Outcomes}

% Upon the completion of a rotation, there are three possibilities:

\begin{enumerate}[nosep]
\item Upon the completion of Rotation 1 or 2, the student and professor agree to
 match permanently: this takes the student out of the rotation process. Then
 the rotation advisor also becomes the student's PhD advisor.
\item After completing Rotation 1, if the student and professor have not agreed
 on a permanent match, the student remains in the rotation program. The student
 must provide a prioritized list of potential advisors with whom they have
 discussed research opportunities. These preferences are used to match each
 student with a research advisor for Rotation 2. Please note that even if both
 the student and the rotation advisor express a desire to continue working
 together for Rotation 2, this is not guaranteed. The availability of rotation
 advisors will ultimately determine whether this arrangement can be made.

 If a student does not secure a successful match upon completing Rotation 2, the
 PhD Graduate Program Director (PGPD) and the Department Chair will meet with the
 student to discuss the next steps to take.
\end{enumerate}

\subsubsection{Rotation Advisors \& Responsibilities}

In Rotation 1, a student shall work with the faculty member who championed their
PhD admission, unless the student and another faculty member mutually agree to
a different arrangement.

Rotation advisors are responsible for evaluating the quality of the student’s
research activities and outcome. This evaluation from Rotation 1 serves as a
component of the student’s grade in CS 6190: Computer Science Perspectives, a
required course.

Rotation advisors are expected to clearly indicate any dissatisfaction with the
student’s research progress, particularly when such concerns contribute to the
decision \textit{not} to form a permanent advisor-advisee relationship. This
feedback can be communicated through various channels, including CS 6190 input,
departmental PhD assessments, and grades for research credits or independent
study.

A student's rotation advisor also functions as their PhD academic advisor. This
means that the advisor for the Fall semester will be the student's Fall
rotation advisor, and the advisor for the Spring semester will be their Spring
rotation advisor.

\hypertarget{phd-qualifying-examination}{
\subsection{Ph.D. Qualifying Examination}
\label{phd-qualifying-examination}
}

The \emph{qualifying examination (quals)} is designed to evaluate a
student's ability to pursue and successfully complete graduate-level
research.

The qualifying examination consists of two parts:
breadth (\Cref{qualifying-examination-breadth-requirement}) and
depth (\Cref{qualifying-examination-depth-requirement}). The \emph{breadth}
portion of the exam helps students obtain broad, graduate-level
knowledge in several major areas of CS. The \emph{depth} portion
of the exam focuses on the student's ability to pursue high-quality
independent research and their ability to effectively communicate about their
research, and requires the student to write a proposal, present it to
their quals committee (\Cref{qualifying-examination-committee}), and then complete a nominally one-semester
independent research project guided by the student's research advisor.

Students who have already passed their qualifying examinations in
\textit{Computer Science or Computer Engineering} at a previous institution may petition the Graduate
Program Committee for an exemption from the UVA CS qualifying exam upon
presentation of acceptable evidence.
The student shall contact the PGPD to initiate such a petition.

% While the completion of the qualifying exam has no hard deadline, the
% expectation is that \ul{by the end of the student's third semester}
% (since the student starts working their PhD advisor)
% the qualifying examination committee should be formed and approved by
% the PGPD, and the form filed with the Graduate Student Coordinator.
% \ul{By the end of the student's fourth semester}
% (since the student starts working their PhD advisor),
% the breadth and depth portions of the qualifying examination should be completed.

\noindent
\textbf{Expected timing.}
\ul{By the end of the student's fourth semester}
(since starting work with their Ph.D. advisor),
the student shall complete the qualification proposal (Phase 1, see \Cref{fig:qual-process}).
\ul{Within six months since completing the qualification proposal},
the student should complete the qualification defense (Phase 2).



% \begin{center}
% \begin{tikzpicture}[node distance=2cm]

%     % Nodes
%     \node (breadth) [startstop] {Breadth Requirement};
%     \node (committee) [process, below of=breadth] {Form Exam Committee};
%     \node (schedule) [process, below of=committee] {Schedule Proposal};
%     \node (proposal) [process, below of=schedule] {Proposal (Phase 1)};
%     \node (exam) [process, below of=proposal] {Exam (Phase 2)};

%     % Box around the depth requirement process
%     \node (depthbox) [group={(committee) (schedule) (proposal) (exam)}] {};

%     % Label for the depth requirement box
%     \node at ($(depthbox.north) + (0, 0.2)$) {Depth Requirement};

%     % Arrows
%     \draw [arrow] (breadth) -- (committee);
%     \draw [arrow] (committee) -- (schedule);
%     \draw [arrow] (schedule) -- (proposal);
%     \draw [arrow] (proposal) -- (exam);

% \end{tikzpicture}
% \end{center}

\begin{figure}[ht]
  \centering
  \includegraphics[width=\textwidth]{figs/qual.pdf}
  \caption{Qualifying Examination Process}
  \label{fig:qual-process}
\end{figure}

\hypertarget{qualifying-examination-committee}{
\subsubsection{Ph.D. Qualifying Examination Committee}
\label{qualifying-examination-committee}
}

% fxl feb 2025: do not weaken the original requirements of >=3 CS profs.
% without req below, coadvisor can be courtesy + 2 CS profs. hence only 2 CS profs on the committee.
This committee must consist of the student's advisor(s) and two additional
Computer Science (CS) faculty members, totaling at least three CS faculty
members.

CS faculty are defined as those with primary or secondary appointments in
Computer Science. Courtesy appointments in CS do not count toward the required
number of CS faculty members on the committee.

The student must form a committee and schedule the qualifying exam at least two
weeks prior to the exam. The committee must have an explicitly designated chair
who will direct meetings and ensure proper procedure. The advisor(s) may not
serve as the chair. Once the committee is formed, the student submits the
Committee Form to the Graduate Program Coordinator.




% \marginpar[hello]{It isn't clear what "well before" means. Also we haven't defined "Phase 1".}
% The chair must be decided at least 1 week before the exam proposal.

\hypertarget{qualifying-examination-breadth-requirement}{
\subsubsection{\texorpdfstring{Qualifying Examination \underline{Breadth} Requirement}{Qualifying Examination Breadth Requirement}}
\label{qualifying-examination-breadth-requirement}
}

To satisfy the breadth requirement, a minimum of one course in
\href{https://docs.google.com/document/d/1UztNiSjNeOGGoJ17v8-ehiHTHC3a2BT5Dy41ZWnVAkI}{any
four of six topical areas} must be taken.
The list of courses that are allowed in each of the six areas can be found on
the CS degree programs page of the Graduate Record (\Cref{phd-graduate-record})
which is updated annually, as well as the department-maintained
\href{https://docs.google.com/document/d/1UztNiSjNeOGGoJ17v8-ehiHTHC3a2BT5Dy41ZWnVAkI/edit?tab=t.0\#heading=h.1m06ls4unej7}{{Google doc}}
which is updated throughout each year.


\hypertarget{qualifying-examination-depth-requirement}{
\subsubsection{\texorpdfstring{Qualifying Examination \underline{Depth} Requirement}{Qualifying Examination Depth Requirement}}
\label{qualifying-examination-depth-requirement}
}

The depth component of the qualifying examination proceeds in two phases. In
Phase 1, the student prepares a written and oral research proposal for an
independent research project and presents this proposal to their evaluation
committee. In Phase 2, the student completes the proposed research, prepares a
written and oral project report, and presents the results to their evaluation
committee. Completing all the proposed tasks is often challenging. We advise
students to clearly explain the difficulties they encountered if they were
unable to complete all tasks. The committee will consider well-supported
explanations. After both phases, the committee provides feedback and determines
the phase outcomes.


The proposal process is intended to assist the student in the formalization of
the research project and to ensure that the student is not undertaking too much
or too little work and that prior work has been properly examined and
understood.

\hypertarget{qualifying-examination-depth-proposal}{
\paragraph{Qualifying Examination Depth Proposal}
\label{qualifying-examination-depth-proposal}
}

For Phase 1 the student prepares a written proposal document and oral presentation.
The student should propose a research project that can be completed in approximately one semester.
The proposal must also include a specific reading list (\Cref{qualifying-examination-proposal-document}).

To help develop the student's ability to propose new research, the proposal must
\emph{not} be a previously submitted or published paper, or an existing paper
that has been reformatted with minor edits (e.g., to recast work already
completed as if it were work to be performed in the future).The proposal may
build upon individual or group research that has already been completed,
provided it represents a meaningful extension of that work to be undertaken
solely by the student. The student must disclose any papers related to the
proposal, including those that have been published, are under review, or have
been posted as preprints, to the committee.

The proposal must have only one author (i.e., the student).
Although, the advisor may help with revisions, the proposal must
ultimately reflect the student's own research activities.

For the oral presentation, the student must schedule a meeting with their qual
evaluation committee. The qualifying examination depth proposal is not open to
the public. During the meeting, the student starts with
a \textasciitilde20-minute presentation about the proposed work, followed by
questions from the committee members about the proposed work and the reading
list.

The student should inform the Department Coordinator of their scheduled proposal
at least one week before their oral presentation.

In preparation for the depth proposal, students are encouraged to communicate
with their committee members to gather advice concerning the proposed topic,
work, reading list, and timeline.

% \st{Phase \st{two} \fxlNote{one} is the formal review of the proposal (written and oral) and reading list; however, this proposal defense is not open to the public.}}


% \st{Phase \st{three} \fxlNote{two} (the qualifying examination final report and presentation) is discussed in sections 3.4.5-3.4.7.}

\hypertarget{qualifying-examination-proposal-document}{
\subparagraph{Qualifying Examination Proposal Document}
\label{qualifying-examination-proposal-document}
}

\textbf{Qualifying Examination Research Proposal.}
The student's proposal document should be sufficient for the committee
to evaluate the research quality of the proposed project. As such, it
should contain the following elements:

\begin{itemize}[nosep]
\item \textbf{Abstract.} An executive summary, no more than one-half page.

\item \textbf{Motivation.} What is the problem and why is it important?

\item \textbf{Hypothesis.} What is the hypothesis of the proposed research?

\item \textbf{Contributions.} What are the main ideas and why do they matter?
In what way are these ideas novel?

\item \textbf{Related work.} What is the relevant prior work and the
state-of-the-art in this area?

\item \textbf{Detailed research plan.} What specific goals or milestones will
be completed during the research project and how will they be
implemented, designed, and evaluated? For projects with a significant
implementation component, give enough details of the features to be
implemented and the experimental setup involved for the committee to
judge the feasibility of the proposed work. For projects with a
significant formal component, give enough details of the formalisms used
(e.g., proposed theorems, proof schemas, and logical frameworks) for the
committee to judge the feasibility of the proposed work. Note that the
research plan must explain how the research is to be evaluated (i.e.,
what are the metrics of success?).

\item \textbf{Summary and Future Work.} A short summary of the above, and
identification of potential future work.

\end{itemize}

In the document, the student must disclose all papers related to the proposal,
including those that are published, under review, or posted as preprints.


\textbf{Qualifying Examination Reading List.} The qualifying exam
proposal should include a reading list that the oral examination may
cover. The student and the advisor should prepare an initial reading list,
which should be included as an appendix in the proposal document.

The reading list should include:

\begin{itemize}[nosep]

\item \textbf{Focus papers}. A small number of papers (typically two or three)
representing the area's state of the art. The student will be expected
to know these papers in detail.

\item \textbf{Background readings}. Typically, a textbook and/or one or two
book chapters or survey papers. The student will be expected to have a
firm command of the material covered in these readings, as shown through
general understanding and the ability to place the work in context.

\end{itemize}

\hypertarget{qualifying-examination-depth-proposal-outcomes}{
\subparagraph{Qualifying Examination Proposal Outcomes}
\label{qualifying-examination-depth-proposal-outcomes}
}

The possible outcomes of Phase 1 are:

\begin{enumerate}[nosep]
    \item The student may proceed to Phase 2. In this case, the committee may
     request amendments or changes, make changes or additions to the reading
     list, set appropriate due dates, and/or indicate weaknesses in the
     proposal that must be addressed in the final report and presentation. The
     committee will also indicate deadlines for any required revisions.

    \item The student may not proceed to Phase 2 and should re-attempt the proposal.
\end{enumerate}


\hypertarget{qualifying-examination-presentation}{
\paragraph{Qualifying Examination Presentation}
\label{qualifying-examination-depth-defense}
}

Phase 2 requires the student to write a project report about their qualifying
exam research and complete an oral presentation.

Before the qualifying exam presentation can be scheduled, the Graduate
Coordinator must certify that the student has fulfilled the \emph
{breadth} requirement.

The student's qualification exam presentation is open to the public. The student
arranges for the Graduate Student Coordinator to publicly announce and
publicize the time, date, place, committee members, and abstract to the CS
Department at least 2 weeks before the qualifying exam.

\hypertarget{qualifying-examination-depth-report}{
\subparagraph{Qualifying Examination Report}
\label{qualifying-examination-depth-report}
}

The student must prepare a written report based on their research project.

Students must email their written report to all committee members at least seven
days before the qualifying exam.

\hypertarget{qualifying-examination}{
\subparagraph{Qualifying Examination Presentation}
\label{qualifying-examination-depth-defense-presentation}
}

The student must present their research project outcomes to their examination
committee. Two hours should be allocated for the qualifying exam. The meeting
begins with the student presenting a \textasciitilde30-minute overview of the
project, followed by Q\&A. The committee will ask questions about the research
project and about the material from the reading list.

To prepare for the presentation, the student should be ready to answer
questions about their depth area in general and their research project
in particular.


\begin{itemize}[nosep]
    \item Students should be able to explain the main idea, conclusions, and
     relevance of any paper included in their report's bibliography. They are
     not expected to know every detail of every paper.

    \item Students should be familiar with the papers from their reading lists,
     as these represent the state of the art in the field. A higher standard of
     understanding will be expected for these papers, and students should be
     prepared for in-depth questioning regarding them.
\end{itemize}


The presentation should use numbered slides. The student should electronically
send the presentation to the committee no later than the day before the
qualification exam presentation.

At the end of the qualification exam, the committee deliberates and decides on an outcome.

\hypertarget{qualifying-examination-outcomes}{
\subsubsection{Qualifying Examination Outcomes}
\label{qualifying-examination-depth-outcomes}
}

After reviewing the student’s final project report and oral presentation, the
examination committee will determine whether the student has passed the depth
portion of the qualifying exam. If the student's performance is not acceptable,
the committee may permit a second attempt, in which case the exam
\emph{must be re-taken within 60 days} (excluding holidays or days when
the University is not in session). A maximum of two attempts is permitted.

\hypertarget{phd-doctoral-committee}{
\subsection{Ph.D. Doctoral Committee}
\label{phd-committee}
}

% https://records.ureg.virginia.edu/content.php?catoid=62&navoid=5418#committee-requirements

%  below updated 4/1/25 by fxl, according to new GR wording
% GR draft
% https://myuva.sharepoint.com/:w:/r/sites/csadminstaff2/Shared%20Documents/Students/Graduate%20Program/Acalog/2025-26%20Updates/PhD%2025-26%20Graduate%20Record%20Updates.docx?d=w7ffdcd949840492ba7b793fb1a1e173a&csf=1&web=1&e=VBARCe

Students should consult ``Committee Requirements'' in the SEAS Academic Rules
section of the Graduate Record (\Cref{phd-graduate-record}). The policies
detailed here complement the SEAS Requirements and clarify their applications
to Computer Science.

The PhD committee (Dissertation Proposal and Dissertation Defense) should be
arranged by the student after the qualifying exam. The committee must consist
of a minimum of five faculty members. Membership must include at least three
Computer Science faculty members, at least one UVA faculty member from outside
the Computer Science department and at least one other member with expertise in
the research area, i.e. 3 CS faculty + 1 UVA faculty (outside CS) + 1 outside
expert. The Department recommends that one of the committee members be an
expert from outside the University.

CS faculty are defined as those with primary or secondary appointments in CS.
Faculty with courtesy appointments in CS do not count toward the required
number of CS faculty members on the committee.

Responsibility of a Faculty Co-Advisor: A faculty co-advisor shares the
responsibility of guiding the student’s academic progress and fostering
interdisciplinary collaboration. The co-advisor is expected to provide
mentoring comparable to that offered to their sole advisees, including regular
guidance on research direction, professional development, and academic
milestones. While co-advisership may not entail immediate GRA support, the
co-advisor should be prepared to provide GRA to the student if funds become
available, in the same spirit as with their other advisees. In addition, the
co-advisor bears responsibility for tracking the student’s PhD progress and
milestone completion, and for contributing meaningfully to the student’s
overall success.


\begin{comment}
  % -----  below are writing before Mar 2025, GR update ----- %
Please refer to the SEAS Committee Requirements in the Graduate Record (\Cref{phd-graduate-record}), which are referred to as “SEAS Requirements” in the text below. The policies detailed here  complement the  SEAS Requirements and clarify their applications to Computer Science.

\textbf{The PhD Advisory Committee}
This committee is responsible for approving a student's dissertation proposal. Its membership must adhere to the criteria set for the Ph.D. Advisory Committee as outlined in the \textit{SEAS Requirements}. Additionally, according to these requirements, this committee must also be part of the student's Ph.D. Defense Committee, which is described in the section below.

\textbf{The PhD Defense Committee}
This committee evaluates a student's Ph.D. dissertation and oral defense. According to \textit{SEAS Requirements}, this committee must include the student’s Ph.D. Advisory Committee which approves the student’s Ph.D. dissertation proposal.  

It must consist of a minimum of five faculty members. Membership must include at least three Computer Science faculty members, at least one UVA faculty member from outside the Computer Science department, and at least one other member with expertise in the research area, i.e. 3 CS faculty + 1 UVA faculty (outside CS) + 1 outside expert.  

The Department recommends that one of the committee members be an expert from outside the University (who should submit a biography in advance to permit prior approval by SEAS). 

Computer Science faculty are defined as those with primary or secondary appointments in Computer Science. Faculty with courtesy appointments in Computer Science do not count toward the required number of CS faculty members on the committee.
\end{comment}

% A PhD student's PhD Committee evaluates the student's PhD dissertation
% and oral defense and must consist of a minimum of five faculty members
% constituted according to the following rules. Membership must include at
% least three Computer Science faculty members, at least one UVA faculty
% member from outside the Computer Science department and at least one
% other member with expertise in the research area,
% i.e. 3 CS faculty + 1 UVA faculty (outside of CS) + 1 outside expert.

% fxl: 3/7/25 Kevin S confirmed that we allowed courtesy faculty to be MS advisor, and offer CS 8999

% fxl: removed 3/6/2025. seems not necessary.
% The Doctoral Advisory Committee form should be completed once the PhD
% Proposal Committee is formed and submitted before the proposal defense.
% Students are encouraged to meet with each potential committee member
% before scheduling the proposal to discuss the scope of their proposal.

\hypertarget{phd-proposal}{
\subsection{Ph.D. Proposal}
\label{phd-proposal}
}

A Ph.D. student must develop a written dissertation proposal, created
under the guidance of the student's advisor(s). This proposal should be
presented to the student's Ph.D. Committee prior to performing extensive
research, to receive early faculty approval of the suitability of the
proposed research.

The Ph.D. student's advisor must have read and approved the dissertation
proposal document and the proposed presentation before the Oral
Examination is scheduled.

% It is recommended that the proposal be completed by the end of the third year.

\noindent
\textbf{Expected timing.}
A Ph.D. student should complete the PhD proposal
\ul{within five years} of starting their work with the PhD advisor.

\hypertarget{phd-proposal-document}{%
\subsubsection{Ph.D. Proposal Document}\label{phd-proposal-document}}

The student's Ph.D. proposal document \emph{should have the same structure
as the Ph.D. Qualifying Examination Proposal Document} (\cref{qualifying-examination-proposal-document}) and
should clearly and unambiguously convey the scope of the work and the
criteria for success. Proposals can also include a section devoted to
the student's work thus far, although this section is not formally
required.

Proposal documents should not exceed 15 single-spaced pages (or 30
double-spaced pages). The bibliography and any appendices (appendices
are not required to be read by the student's committee) are not included
in this page limit. Significant departures from these guidelines must be
approved in advance by the student's proposal committee. The written
proposal document must be submitted to the committee at least two weeks
in advance of the proposal presentation.
% \fxlNote{9/13/25: fxl: TO REVISIT. email comment by Kevin Skadron: "In any case, why is it so long?  For the entire dissertation, we give the committee 2 weeks, so why should the proposal (which is only 15 pages) necessitate 2 weeks?"}
Students are encouraged to
follow the National Science Foundation (NSF) grant proposal formatting
guidelines.

\hypertarget{phd-proposal-presentation}{
\subsubsection{Ph.D. Proposal Presentation}
\label{phd-proposal-presentation}
}

The Ph.D. Proposal Oral Presentation must be  announced publicly at least
\emph{two weeks} in advance through the Graduate Student Coordinator.

The Ph.D. proposal meeting should be scheduled for 2 hours, with the presentation lasting about 30 to 45 minutes, as committee members are expected to have already read the proposal.
After the presentation, the committee
members discuss the proposed work, ask questions, and offer suggestions
or identify required changes. The student initiates the
{\emph{``Dissertation
Proposal and Admission to Candidacy''}} and \emph{``Engineering Dissertation
Proposal Assessment''} \href{https://engineering.virginia.edu/graduate-study/current-graduate-students/academic-planning/forms-graduate-students}{form} the morning of their presentation.

Students are encouraged to provide the committee members with copies of
the slides used in the proposal presentation. Slides can be distributed
in electronic form a few days in advance of the presentation or printed
out and distributed at the presentation itself. Providing
\emph{numbered} slides is a courtesy that helps the committee follow the
presentation and keep track of their comments.

\hypertarget{phd-proposal-outcomes}{%
\subsubsection{Ph.D. Proposal Outcomes}\label{phd-proposal-outcomes}}

After the proposal meeting, there are several possible outcomes:

\begin{itemize}[nosep]
\item
  The proposal is accepted without changes.
\item
  The proposal is not accepted until amendments to the written document
  are made and approved by all committee members.
\item
  The proposal is not accepted, and the student will need to write
  another proposal or modify the proposed work.
\item
  The proposal is not accepted, and the committee indicates that the
  student does not have sufficient research potential to complete a
  dissertation in a timely fashion; in this case, the student is subject
  to dismissal from the program.
\end{itemize}

Once the proposal is accepted, it serves as a binding document for the committee. If the student successfully carries out the work outlined in the proposal, the committee will not reject the student's Ph.D. dissertation due to insufficient progress. The proposal is not binding for the student, who can modify the research plan as necessary. However, there is no guarantee that any research conducted outside of the proposed plan will meet the depth and scope required for the PhD program. Therefore, students who decide to adjust their research plans should consult with their committees.
Significant departures from the proposed work \emph{must} be approved in
advance by the committee.

\hypertarget{phd-dissertation}{
\subsection{Ph.D. Dissertation}
\label{phd-dissertation}
}

The dissertation should convey the research hypothesis, research
paradigm, and research results and then defend the proposition that the
results are valid and correct. The exact form of the dissertation can
vary across topics, but in general a dissertation will include the
following elements:

\begin{itemize}[nosep]
\item
  Presentation of the motivation, hypothesis, and contributions of the
  research.
\item
  Placement of the work in the context of prior art.
\item
  An explanation of how the proposed work was carried out.
\item
  Where applicable, the experimental design of any experiments should be
  provided which provides enough information for the reader to replicate
  the results.
\item
  Conclusions drawn from the work and a discussion of future research
  directions suggested by the project.
\end{itemize}

A dissertation should be a self-contained document. It should not assume
that the reader has read the corresponding proposal, so it should
provide enough context that a reader who has read the proposal can
readily understand how the performed work fulfills the promises in the
proposal. Parts of the proposal can be included in the dissertation.

The written dissertation document must be submitted to the committee at
least two weeks before the oral defense.

\hypertarget{phd-defense}{
\subsection{Ph.D. Defense}
\label{phd-defense}
}

The dissertation defense, which must be announced publicly two weeks in
advance via the Graduate Student Coordinator, is an oral defense before
the student's Ph.D. dissertation committee and other faculty, students,
and visitors. Generally, a defense should be scheduled for at least two
hours to allow audience questions and a post-presentation discussion
by the committee. The student initiates the Ph.D. Dissertation Defense form the morning of their defense.

Before the oral defense, students are encouraged to give committee members
copies of the presentation materials used in the oral defense. These
materials can be distributed electronically a few days in advance of
the presentation or printed and distributed at the defense itself.
Providing \emph{numbered} slides is a courtesy that helps the committee
follow the presentation and keep track of their comments.

A typical defense at UVA CS comprises the following components:

\begin{itemize}[nosep]
\item
  Presentation. It should not exceed 45 minutes. During the presentation, questions from the committee and other members of the audience are expected.
\item
  Q\&A. The committee and other members of the audience ask the candidate questions to answer.
\item
  Deliberation. The committee members have closed-door discussion; the candidate and other audience shall be excused.
\item
  Announcement of the outcome. The chair of the committee notifies the candidate of the decision and any requirements or comments.
\end{itemize}

The Graduate Student Coordinator will send the \emph{Report on
Dissertation or Thesis Final Examination} and \emph{Engineering PhD
Dissertation Assessment forms} to the committee on the morning of the
presentation\emph{.} The student must also complete the \emph{Survey of
Earned Doctorates} and submit the dissertation electronically to
LIBRA at the UVA Library.

\hypertarget{phd-defense-outcomes}{
\subsubsection{Ph.D. Defense Outcomes}
\label{phd-defense-outcomes}
}

Based upon the student's dissertation document and oral exam, the
dissertation committee will either:

\begin{itemize}[nosep]
\item
  approve the dissertation, indicating the student has passed the
  dissertation defense component of the Ph.D. degree, and fill out the
  forms indicating approval, or
\item
  require amendments to the written dissertation and hold the evaluation
  forms until the changes are made satisfactorily, or
\item
  specify significant amendments to the dissertation to be followed by a
  new defense, or
\item
  declare the work unsatisfactory and dismiss the student from the
  program.
\end{itemize}

Students should double-check the completion of their requirements
using the Academic Report option offered by the Student Information
System (SIS) website.

To receive a Ph.D.degree, students must apply for graduation using SIS at
the start of the semester during which they expect to graduate.

\hypertarget{typical-timeline-phd}{
\subsection{Typical Timeline}
\label{typical-timeline-phd}
}

\emph{If entering without a Master's degree:}

\begin{itemize}[nosep]
\item
  First semester: Take graded, graduate-level courses, CS 6190, and CS
  8999 with Rotation One advisor.
\item
  Second semester: Take graded, graduate-level courses and CS 8999 with
  Rotation Two advisor. Match with a research advisor.
\item
  Third semester: Take remaining coursework. Form a qualifying
  examination committee. Work with your research advisor
  (CS 8999). Work as half-time TA and take CS 8897 (Graduate Teaching
  Instruction).
\item
  Fourth semester: Work as half-time TA and take CS 8897 (Graduate
  Teaching Instruction). Continue working with your research advisor (CS
  8999). Take Qualifying Examination to fulfill the depth requirement and
  certify completion of the breadth requirement.
\item
  Fifth semester and beyond: If the Qualifying Examination has been
  successfully completed, then take CS 9999; otherwise, take CS 8999.
  Prepare and defend the Dissertation Proposal. Execute the work
  proposed for the dissertation. Finally, write the dissertation and
  pass an oral defense.
\item
  Submit the dissertation electronically to LIBRA and submit the survey of
  earned doctorates.
\end{itemize}

\emph{If entering with a Master's degree:}

\begin{itemize}[nosep]
\item
  Present appropriate documentation (e.g., transcript) for your Master's
  degree in a computing field. SEAS will make a ``bulk transfer'' of 24
  credits (regardless of the actual number of credits taken for the
  Master's degree). Before choosing your two additional CS courses
  required to be taken at UVA, verify that the graduate courses taken
  for your Master's degree also fulfill our Qualifying Examination
  breadth requirement (consult the Ph.D. Graduate Program Director). If
  the breadth requirement is not completely satisfied by the graduate
  courses taken elsewhere and used to generate the 24-credit ``bulk
  transfer,'' then choose courses as necessary to fulfill the breadth
  requirement. You must complete a minimum of 6 credits taken at UVA.
\item
  If you passed a Ph.D. Qualifying Examination \emph{in Computer Science}
  at a previous institution, you may petition the Graduate Study
  Committee to waive the depth portion (but not the breadth portion) of
  the CS quals and present appropriate documentation (e.g., a letter
  from the previous institution).
\item
  If you took an equivalent course to our CS 6190 (Perspectives), submit the
  appropriate documentation (e.g., transcript) and you will be exempted
  from our CS 6190 (3 credits) requirement.
\item
  If you are entering as a third-year (or later) PhD student, you can
  request an exemption from CS 6190 from the PhD Graduate Program
  Director.
\item
  As soon as practical, form your Ph.D. Doctoral Committee
  and file the appropriate form. The Graduate Coordinator will help with this.
\item
  Prepare and defend your Ph.D. Dissertation Proposal. File assessment
  paperwork when satisfactorily completed.
\item
  Complete the work defined in your proposal and then write and defend
  your written dissertation in an oral presentation. File assessment
  form when satisfactorily completed.
\item
  Submit your dissertation electronically to LIBRA and take the survey of earned doctorates.
\end{itemize}

\hypertarget{first-year-fellowship-gras-and-gtas}{
\subsection{First-Year Fellowship, GRAs and GTAs}
\label{first-year-fellowship-gras-and-gtas}
}

A Ph.D. student is usually employed by the department as a
\emph{Graduate Teaching Assistant (GTA)}, a \emph{Graduate Research
Assistant (GRA)}, or via a \emph{First-Year Fellowship}. The First-Year
Fellowship provides student support for required duties such as
classwork, research, and rotations; as such, it represents taxable income
(unlike other Fellowships which impose no duties). As per SEAS policy, a
funded student is not allowed to have outside employment without
permission from the Computer Science Chair and the SEAS Director of
Graduate Programs. Full-time graduate students must not unilaterally
accept internships without the prior approval of their advisor.

With some exceptions, all full-time graduate students with departmental
funding must sign up for at least 12 credit hours for the fall and spring semesters. CS 6190, CS 6993, CS 7993, CS 7995, CS 8987, CS 8999,
CS 9897, and CS 9999 all count toward the full-time requirement.

Stipends increase after successful completion of the PhD qualifying
examination, and again after successful completion of the PhD proposal
presentation.

\hypertarget{first-year-fellowship}{
\subsubsection{\texorpdfstring{First-Year Fellowship}{First-Year Fellowship}}
\label{first-year-fellowship}
}

See \Cref{first-year-phd-rotation-program}.

\begin{comment}
Entering PhD students (excluding transfers into CS from another
department) are typically funded in their first year (fall, spring,
summer) by a First-Year Fellowship. During this time, students are
expected to attend classes, complete any English as a Second Language
(ESL) course requirements, conduct research with the rotation
advisor(s), and eventually match with a research advisor (see rotation
discussion in \cref{first-year-phd-rotation-program}).
\end{comment}

\hypertarget{graduate-teaching-assistant-gta-responsibilities}{
\subsubsection{Graduate Teaching Assistant (GTA) Responsibilities}
\label{graduate-teaching-assistant-gta-responsibilities}
}

Graduate Teaching Assistants (GTAs) are important members of the
department's professional teaching staff. GTA responsibilities for each
course are assigned by the instructor. Duties typically include grading,
proctoring laboratory sections, holding office hours and help sessions,
attending class, reading instructional materials, completing
assignments, answering email or forum questions, and tutoring students
in need of additional help. GTAs may also contribute study questions or
examination questions at the discretion of the instructor. GTA
assignments are made by the Graduate Student Coordinator early in the
semester (and may change early in the semester) and are accompanied by
an expected number of hours the GTA should devote to each course. GTAs
without a firm grasp of course concepts should obtain guidance from the
instructor or request a change in course assignments from the CS staff
when given the course assignment. Students concerned that specific
duties of the GTA are inappropriate/off-topic or require more effort
than allocated may seek resolution through the course instructor, their
advisor, or the Graduate Student Ombudsman.

PhD students are \emph{required} to serve as GTAs as a component of
their degree. A GTA must sign up for 3 credit hours of CS 8897/9897
(Graduate Teaching Instruction), using the specific section assigned to
the instructor, for each 10 hour/week segment. PhD candidates (those who
have already successfully completed the Qualifying Examination should
enroll in CS 9897; otherwise use CS 8897. GTAs assigned to multiple
courses should split the amounts among those courses at their
discretion, noting that it is not possible to sign up for fractional
credit hours. Completion of the GTA portion of the PhD requirement is
signified by having accumulated 12 credit hours of CS 8897/9897,
typically over Year 2 and Year 3 of PhD studies.

GTAs are representatives of the Department and the University. As such,
they are expected to behave with professional courtesy and politeness in
all their official communications and activities, including handling
student questions in a polite, constructive, inclusive, and accurate
manner. GTA conduct is governed by the general
\href{https://provost.virginia.edu/academic-policies/conflict-interest-faculty}{conflict
of interest policies} of UVA.

The period of GTA engagement begins at the start of each semester and
lasts until final grades are submitted to the registrar. GTAs
should be reliable in all their duties. Non-emergency absences from
scheduled duties within that time must be approved by the course
instructor and PGPD. As an example, GTAs may not depart before final
exams are graded and course grades are submitted without the advance
approval of their instructor and the PGPD.

\hypertarget{graduate-research-assistant-gra-responsibilities}{
\subsubsection{Graduate Research Assistant (GRA) Responsibilities}
\label{graduate-research-assistant-gra-responsibilities}
}

Ph.D. students receiving research funding through one or more professors
are called Graduate Research Assistants (GRAs). Much of a typical Ph.D.
student's academic tenure is spent as a GRA. GRAs and advisors are
colleagues in research and the employer-employee relationship is rarely
visible as they work together to engage in a research project. While a
GRA officially a 20 hour/week position, with respect to funding, success
in graduate school requires substantially more effort. For instance, a
student is expected to devote at least 3 hours/week outside the
classroom for each academic credit. In general, a GRA is expected to
work as directed by his or her research advisor. However, a student who
is concerned that specific duties are inappropriate, or off-topic may
seek resolution with their research advisor or the Ph.D. Graduate Program
Director (PGPOD) or the Graduate Student Ombudsman. GRAs are expected to be
physically present from the first day of classes until the last day of
exams. All absences must be approved by the research advisor and must
conform to the School of Engineering's policy regarding graduate student
leave.

\hypertarget{summer-support}{
\subsubsection{Summer Support}
\label{summer-support}
}

Students may wish to gain direct experience with government or
industrial research through summer internships (during any summer). A
student interested in an internship should get the approval of his/her
advisor. Graduate-level summer internships often lead to a publication,
provide external committee members and help in the student's evaluation
of possible careers. Research advisors, the SEAS Center for Engineering
Career Development, and the UVA Career Center can help find suitable
summer employment. PhD students who do not pursue internships are
typically supported over the summer as GRAs (funded by their advisors)
and must register as full-time students (6 credits in the summer). Students
who have passed the Ph.D. Qualifying Exam enroll in CS 9999; those who have not enroll in CS 8999.
