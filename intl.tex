% !TeX root = main.tex

\hypertarget{international-students}{%
\section{International Students}\label{international-students}}

According to UVA policy, international students should directly consult the International Studies Office (ISO) regarding any matters affecting their legal status.

\href{https://issp.virginia.edu/}{{International Studies Office
Website}}

\hypertarget{full-time-status}{%
\subsection{Full-Time Status}\label{full-time-status}}

Please consult with the International Studies Office regarding full-time
status requirements: \href{https://issp.virginia.edu/enrollment}{{https://issp.virginia.edu/enrollment}}

\hypertarget{english-language-proficiency-assessments-written-and-oral}{%
\subsection{English Language Proficiency Assessments (Written and
Oral)}\label{english-language-proficiency-assessments-written-and-oral}}

The Center for American English Language and Culture (CAELC) administers
the University of Virginia English Language Proficiency Exam (UVELPE).
For more information regarding UVELPE, refer to
\url{https://caelc.virginia.edu/assessment.}

The UVA Computer Science Department believes that University-provided
ESL (English as a Second Language) courses are provided solely for the
the benefit of the student in both academics and future employment

Please refer to the SEAS Graduate Record for language requirements:
\url{http://records.ureg.virginia.edu/content.php?catoid=57\&navoid=5188\#esl-courses}

According to CS department policy, students must comply with the ESL
recommendations. UVELPE tests and ESL requirements are Engineering School
requirements and cannot be waived by the CS Department.

\hypertarget{curricular-practical-training-cpt}{%
\subsection{Curricular Practical Training (CPT)}\label{curricular-practical-training-cpt}}

International students should contact the International Studies Office
(ISO) when considering a summer internship.

UVA CS Master's and Ph.D. students pursuing CPT should do the following:
in the Fall semester \emph{after} the summer internship, register for
one credit hour of CS 6890 (Industrial Applications) with their academic
or research advisor.

The general requirement of the CS 6890 course is to report on (1) when,
where, and with whom the internship was served, (2) what was learned and
what new insights were gained, and (3) how the internship experience is
expected to assist future academic or employment pursuits. The details
and specific requirements of the course are supervised by the advisor.
CS 6890 is evaluated as Satisfactory or Unsatisfactory (S/U) and does
not count for any of the graduate degree requirements.

In the rare event that a student completes a CPT internship in the Fall
or Spring, the student may enroll in CS 6890 in the same semester \emph{or} the
subsequent semester.

\hypertarget{optional-practical-training-opt}{%
\subsection{Optional Practical Training (OPT)}\label{optional-practical-training-opt}}

\emph{Optional Practical Training (OPT)} is available after graduation
for students on an F-1 visa. Contact the International Studies Office
(ISO) for more details.