
\hypertarget{appendix-c-cs-policy-on-defunding-a-gra}{%
\section{CS Policy on Ph.D. Program Termination}\label{appendix-c-cs-policy-on-defunding-a-gra}}

Once a Ph.D. student has matched with a faculty research advisor 
(referred to as \underline{``advisor''} in the rest of this policy), 
that event establishes a two-way obligation such that (1) the student agrees to
make continuous and satisfactory progress toward his or her degree, and
(2) the advisor agrees to provide mentoring and GRA support
throughout the student's tenure, 
assuming the student is making satisfactory progress. 
A student's performance may be deemed unsatisfactory for reasons such as:

\begin{itemize}
\item
  An individual graduate course has a grade below C
\item
  Graduate GPA is lower than the required B average
\item
  As determined by the advisor or Ph.D. Graduate Program Director (PGPD):

  \begin{itemize}
  \item
    Student is substantially late finishing the required coursework
  \item
    Student is not making progress on program milestones such as the
    qualifying exam, Ph.D. proposal, or dissertation
  \item
    Student is not making adequate progress on research, is not
    producing papers of apparent publishable quality, or repeatedly
    fails to meet reasonable milestones set out by the advisor
  \item
    Student has significant difficulty working within a research group
    (i.e., working collegially with peers)
  \end{itemize}
\item
  Student has significant difficulty with oral and/or written
  communications not remedied by ESL courses
\item
  Violation of the policy on GTA/GRA leave
\item
  Violation of the policies on acceptable use of CS and/or UVA computing
  equipment
\item
  Other specific criteria as predefined by the advisor and approved by
  the PGPD.
\end{itemize}

Special situations such as long-term illness or parental leave are
covered by other departmental and/or Provost policies.

If a student fails to maintain continuous and satisfactory performance, 
and if as a result the advisor 
assesses that the student is not capable of finishing a PhD and/or no longer wishes to advise the student, then, in accordance with the Provost's policy
on Graduate Assistantships PROV-001 (available at
\href{https://uvapolicy.virginia.edu/policy/PROV-001}{{https://uvapolicy.virginia.edu/policy/PROV-001}})
and SEAS rules, the following procedures should be followed.

\begin{enumerate}
    \item 
The advisor must meet with the student to (a) identify what
aspects of the student's performance are unsatisfactory, and (2) explain
in writing what changes must occur and on what timeline (minimum of two
months) for the student's performance to be once again considered
satisfactory. If the student wishes to continue working with this
advisor, they must be given adequate time to improve their performance
and meet the advisor's set of milestones.

\item The advisor is obligated to report any such ongoing situation
to the PGPD each time the department
conducts its PhD student assessment. 
The advisor may report their concerns to the PGPD separately from the PhD assessment process.

\item 
% The advisor is obligated to signal unsatisfactory performance via course grades. 
% They should submit a grade of ``U'' (unsatisfactory) for one or more
% current-semester research courses (e.g., CS 9999); 
% if the advisor supervises the student's independent study (e.g., CS 6993 and CS 7993), 
% the advisor should assign grades that reflect the student's unsatisfactory performance.
The advisor is obligated to signal unsatisfactory performance via course grades, 
including assigning grades that reflect the student's unsatisfactory performance
to the student's research credits (e.g., CS 9999) and/or to the student's independent study (e.g., CS 6993 and CS 7993). 
If a student's performance has been borderline and an advisor needs more time to
determine a grade (e.g. if a U is warranted for CS 9999), 
an incomplete (IN) may be assigned temporarily to give the student time to improve their performance. This
may be done in conjunction with step 1 above.

% FL 3/31/25 notes -- 
% (1) SEAS has its standard for “a reasonable notice”, which we will meet. 
% (2) The handbook also should urge our faculty to give as much early warning signs as possible. 

\item If mentoring attempts by the advisor (and optionally the PGPD) as described above are not successful, then the following procedures are invoked:


\begin{itemize}
\item
  The advisor notifies the student and the PGPD of the advisor's intent to terminate their advisor-advisee relationship
  and the advisor's intended date of termination 
  (end of the current semester is strongly encouraged).
\item
  The advisor provides the student and PGPD with a termination letter, explaining the reasons for the termination. This is a separate
  notification (and later by at least two months) from the first
  notification (warning letter) described in \#1 above.

\item
  GRA funding from the current advisor shall 
  continue for the \emph{longer} of (1) two months or (2) the remainder
  of the current term (Fall, Spring, Summer).  
For terminating in the \textit{Spring} semester, 
the termination letter should be delivered no later than \textit{Mar 15th} of the same year; 
For terminating in the \textit{Fall} semester, 
the termination letter should be delivered no later than \textit{Oct 15th} of the same year. 


% fxl, Mar 2025
% Termination notification date        |    Earliest date of RA funding stop
% Jan 1 – Mar 14                                                      Two months after the notification date
% Mar 15 -- ??                                             The day before the Fall semester start (i.e. the Arrival date)                
% Fall semester start – Oct 15                          Two months after the notification date
% Oct 15 – Dec 31                                     End of the next Spring semester (i.e. the last day of examination)

 
  
% \item
%   If desired by the student, a meeting is arranged with the advisor and
%   PGPD (and optionally others who might be helpful) to discuss the
%   issue.
% \item
%   If, after discussion, the decision is to proceed with defunding, then
%   the student may (a) exit the program immediately (mid-semester, and
%   funding will be withdrawn immediately), or (b) finish the current
%   semester (e.g., to complete courses or complete a master's degree, or
%   (c) search for a new PhD advisor.
\item If desired by the student, they may petition to search for a new Ph.D. advisor. 
A petition should be submitted to the PGPD no later than one month after the advisor's notification of termination; 
it will be reviewed for approval by the CS graduate committee. 


\end{itemize}

\end{enumerate}

% commented out by fxl, 2/23/2025, as suggested by Sandhya. 
% It should also be noted that defunding a GRA can have serious
% consequences, especially for those on an F-1/J-1 visa. For more details,
% consult the International Studies Office (ISO).

% Mar 2025 addition by fxl
\textbf{Termination of Ph.D. Program without a permanent advisor} 
If a student fails to establish a permanent advisor-advisee relationship by the end of their first year, and there is insufficient evidence showing that the student can successfully complete PhD in the department, they will be dismissed from the Ph.D. program.

In such a case, a student may submit a petition to seek a new Ph.D. advisor. This petition must be submitted to the PGPD within one month of the match outcome from the student's second rotation. 
The petition will be reviewed by the CS graduate committee, and approval is not guaranteed.

Deviations from this policy due to exceptional circumstances will be
handled on a case-by-case basis.